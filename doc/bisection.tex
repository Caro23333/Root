\subsection{前提准备}

根据事实,我们自然地考虑以二分法为主体求解,配合一系列对区间的其他划分、调整操作来处理特殊情况。

\begin{problem}[对可能存在的实根作界的估计]~
	
	由于我们计划以二分法为主体来进行求解,需要先确定实根的上下界。
	
	\contribution{吕敬一}
	
\end{problem}

\begin{theorem}[多项式实根的界]~
	
	多项式方程 $a_nx^n+\dots+a_1x+a_0 = 0$ 如果存在实根,则实根必在区间 $[-bound, bound]$ 中,其中
	$$
	bound = 1 + \max_{i = 0}^{n - 1} \left|\frac{a_i}{a_n}\right|
	$$
	
	\contribution[摘自维基百科\footnotemark]{吕敬一}
\end{theorem}
\footnotetext{\url{https://en.wikipedia.org/wiki/Geometrical_properties_of_polynomial_roots}}

\begin{proof}[多项式实根的界]~
	
	设 $x_0 \in \mathbb{R}^*$ 是原多项式的一个实根,$|x_0| \le 1$ 的情形自不必证。下设 $|x_0| > 1$,则有
	$$
	a_nx_0^n + a_{n - 1}x_0^{n - 1} + \cdots + a_0 = 0
	$$
	
	即
	$$
	\begin{aligned}
	|a_nx_0^n| & = \left|\sum_{i = 0}^{n - 1} a_ix^i\right| \\
	& \le \sum_{i = 0}^{n - 1} |a_ix_0^i| \\
	& \le \max_{i = 0}^{n - 1} |a_i| \sum_{i = 0}^{n - 1}|x_0^i| \\
	& = \max_{i = 0}^{n - 1} |a_i| \dfrac{|x_0|^n - 1}{|x_0| - 1} \\
	& < \max_{i = 0}^{n - 1} |a_i| \dfrac{|x_0|^n}{|x_0| - 1}
	\end{aligned}
	$$
	
	从而
	$$
	|x_0| - 1 < \dfrac{\max_{i = 0}^{n - 1} |a_i|}{|a_n|}
	$$
	
	即
	$$
	|x_0| < 1 + \max_{i = 0}^{n - 1} \left|\frac{a_i}{a_n}\right|
	$$
	\qedhere
	
	\contribution{吕敬一}
	
\end{proof}

在求得实根的界之后,设计了如下算法来求解实根。

\subsection{算法流程}

\begin{algorithm}[朴素二分法]~
	
	\textbf{步骤1.} 将区间 $[-bound, bound]$ 作为初始区间插入求解队列 q 中。
	
	\textbf{步骤2.} 若已经找到的实根数与多项式次数相同,或队列 q 为空,则跳转到 \textbf{步骤9}。否则,取出队列 q 最前端的区间 $[l, r]$ ,若区间未空则重新执行 \textbf{步骤2},否则计算其端点的函数值 leftVal 与 rightVal ,并与0进行比较(此处考虑容差)。
	
	\textbf{步骤3.} 若左端点或右端点函数值为 0,则调用 \verb|gb_getRepetition()| 计算该点处的重根数并将其加入答案列表中,然后调用 \verb|gb_findNotZero()| 缩小区间,使得端点函数值不为 0.
	
	\textbf{步骤4.} 若区间长度过短(小于容差值)则取其中点 $m$ 代入函数判断是否是 0,若是则同样计算重根数并加入答案,而后跳转到 \textbf{步骤2}.
	
	\textbf{步骤5.} 若区间两端函数值异号,则执行\textbf{步骤6、7}。否则,执行 \textbf{步骤8}.
	
	\textbf{步骤6.} 区间两端函数值异号,则根据介值定理,区间内必有实根存在,尝试找到其中一个根 $res$ 。记当前正在处理的区间为 $I = [l, r]$,取其中点 $m$,若 $m$ 处函数值为0或区间过短(小于容差),则记 $res$ 为 $m$ 并进入下一步;否则,若 $l$ 点和 $m$ 点函数值异号,则将 $I$ 置为 $[l, m]$ 并重新执行这一步;否则,若点 $r$ 和点 $m$ 函数值异号(这必然成立),则将 $I$ 置为 $[m, r]$ 并重新执行这一步。
	
	\textbf{步骤7.} 计算点 $res$ 处的重根数并将其加入答案列表,然后调用 \verb|gb_findNotZero()| 找到 $res$ 前后最近的函数值不为 0 的实数 $r_0, l_0$,并将 $[l, r_0]$ 和 $[l_0, r]$ 插入到队列 q 末尾。而后重新回到步骤2.
	
	\textbf{步骤8.} 在当前迭代深度不过大的情况下,将当前处理的区间等分为8段,并依次加入到队列q末尾。而后重新回到 \textbf{步骤2}.
	
	\textbf{步骤9.} 对于已求出的实根,考虑到过程中的一系列误差可能性,对根进行进一步去重。将答案列表从小到大排序,记上一个保留的根为 $x$,$x$ 初始为最小的实根。从头遍历,若当前实根减去 $x$ 大于容差,则将 $x$ 更新为当前实根;否则将 $x$ 的重复数加上当前实根的重复数,并设法删除当前实根。
	
	\textbf{算法结束。} 此算法由吕敬一进行了代码实现。
	
	\contribution{吕敬一}
	
\end{algorithm}

\subsection{数值验证}

由于本算法存在明显的正确性问题,暂不对其进行理论证明。关于数值验证涉及的具体方法,见本文第五节:数值验证的方法。

对算法1进行了数值验证,注意到三条主要问题:

(1) 偶数次重根会被直接忽略,或者在某些情况下在实际根附近求出若干不精确的根。初步判断为二分法的结构性缺陷(参考引理2),需要后续改进算法。

\begin{example}[]~
	
	当多项式为
	$$
	\begin{aligned}
	-0.0000000201268990957616007532531384642895955217056780384154990315 & + \\ 0.0000021236101838969952624312006322915280520646774675697088241577x & + \\ -0.0000503647104449539523523919626324385490079293958842754364013672x^2 & + \\ -0.000570193171519725052441562862526325261569581925868988037109375x^3 & + \\ 0.0156122311833811790171555600181818590499460697174072265625x^4 & + \\ -0.06494494310663577463227369435116997919976711273193359375x^5 & + \\ -0.1290991197533018775001067979246727190911769866943359375x^6 & + \\ 0.71293559156111252494980590199702419340610504150390625x^7 & + \\ 0.96451662622768574717468936796649359166622161865234375x^8 & + \\ -2.00802000715429596056083028088323771953582763671875x^9 & + \\ -3.3939905177597129437572220922447741031646728515625x^{10} & + \\ 0.3866014721655073316242123837582767009735107421875x^{11} & + \\ 2.5719003370867863367266181739978492259979248046875x^{12} & + \\ 
	x^{13}
	\end{aligned}
	$$ 
	时(精确实根为 -1.43822, -0.910655, -0.844453, -0.743001, -0.600247, -0.0570855, 0.0187824, 0.0187824, 0.106847, 0.197466, 0.197466, 0.490085, 0.992328 ),所求出的根为 -1.43822, -0.910654, -0.844448, -0.743005, -0.600246, -0.0570885, 0.106844, 0.49008, 0.992327,不难发现各出现了两次的 0.0187824 和 0.197466 被漏掉了。

	当多项式为
	$$
	(x + 0.88971571859204046095470630461932159960269927978515625)^{10}
	$$
	时(精确实根为 -0.88971571859204046095470630461932159960269927978515625,重根10次),所求出的实根为-0.930381和-0.849622,结果精度非常低。 
	
	\contribution{吕敬一}

\end{example}

(2) 某些根与实际根的误差超过了容差。有可能是代码实现问题,也有可能是多项式求值时容差选取不够优秀,还有可能是浮点数自身的误差导致。

\begin{example}[]~
	
	当多项式为
	$$
	\begin{aligned}
	-0.0000058064403473063785275599599233764536165836034342646598815918 & + \\ 0.0002874957679957463956531282800455073811463080346584320068359375x & + \\ -0.00613773044503206598176969777114209136925637722015380859375x^2 & + \\ 0.07348729113091197195917203544013318605720996856689453125x^3 & + \\ -0.53638136681462744714821155866957269608974456787109375x^4 & + \\ 2.420421284825134744522756591322831809520721435546875x^5 & + \\ -6.490042806599969793523996486328542232513427734375x^6 & + \\ 9.2100965283636497815678012557327747344970703125x^7 & + \\ -5.249481342694291896577851730398833751678466796875x^8 & + \\ 
	x^9
	\end{aligned}	
	$$
	时(精确实根为7次重根的0.144122和两次重根的2.12031),求出的实根为0.145712,其与0.144122的误差超过了预设的容差 $10 ^ {-5}$。
	
	\contribution{吕敬一}
	
\end{example}

(3) 在算法1的步骤8中进行了将区间等分并递归求解的操作。这一操作是为了在两端函数值同号的区间中找到两端函数值异号的区间以进行二分,但实际上产生了很多不必要的区间,造成了巨大的时间开销,程序运行缓慢。
