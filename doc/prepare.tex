\subsection{定义}

首先给出接下来会用到的一些定义:

\begin{definition}[邻域]~

	\textbf{$x_0 \in \mathbb{R}$ 的邻域 $U(x_0, \delta) = (x_0 - \delta, x_0 + \delta)$} 
	
	\textbf{$x_0 \in \mathbb{R}$ 的去心邻域 $\mathring{U}(x_0, \delta) = (x_0 - \delta, x_0) \cup (x_0, x_0 + \delta)$}
	
\end{definition}

\begin{definition}[多项式的取模与整除]~

	对于多项式 $f(x), g(x)$ ,记 $f(x)$ 对 $g(x)$ 取模的结果为 $\textrm{rem}(f(x), g(x))$ ,则 $\textrm{rem}(f(x), g(x))$ 是满足如下条件的唯一多项式( $h(x)$ 也是多项式):
	$$
	\left\{
	\begin{aligned}
		& f(x) = g(x)h(x) + \textrm{rem}(f(x), g(x)) \\
		& \deg \textrm{rem}(f(x), g(x)) < \deg g(x)
	\end{aligned}
	\right.
	$$
	
	如果 $\textrm{rem}(f(x), g(x)) = 0$ ,则称\textbf{ $f(x)$ 被 $g(x)$ 整除}或\textbf{ $f(x)$ 包含因子 $g(x)$ }。
	
\end{definition}

\begin{definition}[多项式的最大公因式]~
	
	如果多项式 $f(x)$ 和 $g(x)$ 都被多项式 $h(x)$ 整除,则称\textbf{ $h(x)$ 是 $f(x)$ 和 $g(x)$ 的公因式}。在 $f(x)$ 和 $g(x)$ 的公因式中次数最高且最高次项系数为1的被称为\textbf{ $f(x)$ 和 $g(x)$ 的最大公因式},记为 $\gcd(f(x), g(x))$ 。两个多项式的最大公因式唯一且非零。
	
\end{definition}

\subsection{基本事实}

在研究过程中,我们注意到如下基本事实:

\begin{fact}[介值定理]~
	
	对于 $\mathbb{R}$ 上的连续函数 $f(x)$,若 $f(a)f(b) < 0(a < b)$,则 $(a, b)$ 中存在 $f(x)$ 的零点。
	
\end{fact}

\begin{fact}[代数基本定理的推论]~

	任一实系数 $n$ 次多项式至多有 $n$ 个实根(含重根)。

\end{fact}

\begin{fact}[带余项的泰勒公式]~
	
	设 $f(x)$ 在 $x_0$ 的某个邻域内存在 $n$ 阶导数,则当 $x \rightarrow x_0$ 时,
	$$
	\begin{aligned}
	f(x) = \sum_{k = 0}^{n} \frac{f^{(k)}(x_0)}{k!} (x - x_0)^k + o((x - x_0)^n) \\
	f(x) = \sum_{k = 0}^{n - 1} \frac{f^{(k)}(x_0)}{k!} (x - x_0)^k + \frac{f^{(n)}(\xi)}{n!} (x - x_0)^n
	\end{aligned}
	$$
	
	其中 $\xi$ 在 $x_0$ 和 $x$ 之间。
	
\end{fact}

\begin{fact}[最大公因式的基本性质]~

	以下 $f(x), g(x), h(x)$ 为多项式,$a (a \neq 0)$ 为 $f(x)$ 的最高次项系数,$k$ 为任意非零实数。
	$$
	\begin{aligned}
	\gcd(f(x), kg(x)) & = \gcd(f(x), g(x)) \\
	\gcd(f(x), g(x)) &= \gcd(f(x), g(x) + h(x)f(x)) \\
	\gcd(f(x), 0) & = \frac{f(x)}{a} \\
	\gcd(f(x)h(x), g(x)h(x)) = h(x) \gcd(f(x), g(x))
	\end{aligned}
	$$

\end{fact}

\begin{fact}[多项式的微分性质]~

	任意多项式都是 $\mathbb{R}$ 上的连续函数;
	
	任意多项式的导数还是多项式;

	若某多项式在 $x = x_0$ 处取极小值,则存在 $\delta > 0$ 使得 $\forall x \in (x_0 - \delta, x_0), f'(x) < 0$ 且 $\forall x \in (x_0, x_0 + \delta), f'(x) > 0$ 。极大值情况同理。
	
\end{fact}

\subsection{引理}

并给出了如下引理:

\begin{lemma}[多项式重根数与高阶导数值的关系]~
	
	若多项式 $f(x)$ 有实根 $x_0$ ,则 $f(x)$ 在 $x = x_0$ 处的重根数为 $r (r \le \deg f)$ 是 $f^{(i)}(x_0) = 0, \ (i \in \{1, 2, \cdots, r - 1\})$ 且 $f^{(r)}(x_0) \neq 0$ 的充分条件。
	
	\contribution{吕敬一}
	
\end{lemma}

下文朴素二分法与Sturm-Newton算法中gb\_getRepetition()函数均为该引理的直接应用,其原理不再赘述。

\begin{proof}[多项式重根数与高阶导数值的关系]~
	
	据假设,多项式 $f(x)$ 含有因子 $(x - x_0)^r$ ,不妨设 $f(x) = (x - x_0)^r g(x)$ ,其中 $g(x)$ 不含因子 $(x - x_0)$ 。考虑 $f(x)$ 的 $1, 2, \cdots, r$ 阶导数:
	$$
	\begin{aligned}
		f^{(1)}(x) & = r(x - x_0)^{r - 1}g(x) + (x - x_0)^rg^{(1)}(x) \\
		f^{(2)}(x) & = r(r - 1)(x - x_0)^{r - 2}g(x) + 2r(x - x_0)^{r - 1}g^{(1)}(x) + (x - x_0)^rg^{(2)}(x) \\
		f^{(3)}(x) & = r(r - 1)(r - 2)(x - x_0)^{r - 3}g(x) + 3r(r - 1)(x - x_0)^{r - 2}g^{(1)}(x) + 3r(x - x_0)^{r - 1}g^{(2)}(x) \\
		& + (x - x_0)^rg^{(3)}(x) \\
		& \vdots \\
		f^{(r - 1)}(x) & = \sum_{i = 0}^{r - 1} \binom{r - 1}{i} (x - x_0)^{r - i}g^{(r - 1 - i)}(x) \prod_{j = 0}^{i - 1} (r - j) \\
		f^{(r)}(x) & = \sum_{i = 0}^{r} \binom{r}{i} (x - x_0)^{r - i}g^{(r - i)}(x) \prod_{j = 0}^{i - 1} (r - j) \\
	\end{aligned}		
	$$
	
	观察发现 $f^{(1)}(x), f^{(2)}(x), \cdots, f^{(r - 1)}(x)$ 中每一项都被 $(x - x_0)$ 整除,从而 $f^{(1)}(x) = f^{(2)}(x) = f^{(r - 1)}(x_0) = 0$ 。而 $f^{(r)}(x)$ 中恰有一项 $(x - x_0)^0g(x)r!$ 不含因子 $(x - x_0)$ ,故 $f^{(r)}(x_0) \neq 0$ 。
	\qedhere
	
	\contribution{吕敬一}
	
\end{proof}

\begin{lemma}[无重根多项式与其导函数的公因子]~
	
	对于无重根多项式 $f(x)$,及其导函数 $f'(x)$,有其最大公因子为 $1$,即 $\gcd(f(x), f'(x)) = 1$

	\contribution{潘佳奇}
	
\end{lemma}

\begin{proof}[无重根多项式与其导函数的公因子]~

	对于 $f(x)$ 得其分解 $f(x) = (x - x_1) (x - x_2) \dots (x - x_n)$,则 $f(x)$ 只可能有 $(x - x_i)$ 形式的因子。

	对于 $f'(x)$,有
	$$
	f'(x) = \sum_{i=1}^{n} -x_i (x - x_1) \cdots (x - x_{i - 1}) (x - x_{i + 1}) \cdots (x - x_n).
	$$

	假设 $f'(x)$ 被 $(x - x_i)$ 整除,则有
	$$
	f'(x) \equiv -x_i(x - x_1) \cdots (x - x_{i - 1}) (x - x_{i + 1}) \cdots (x - x_n) \pmod{(x - x_i)}.
	$$

	但是因为 $f(x)$ 没有重根,$(x - x_1) \cdots (x - x_{i - 1}) (x - x_{i + 1}) \cdots (x - x_n)$ 不被 $(x - x_i)$ 整除,与 $f'(x)$ 被 $(x - x_i)$ 整除矛盾。

	所以 $\gcd(f(x), f'(x)) = 1$。
	\qedhere

	\contribution{潘佳奇}
	
\end{proof}
