在小组成员的算法设计中,三个人各采用了三种算法。

吕敬一采用了 Sturm 区间分割达到足够小区间以使用 Newton 迭代求根的方法。

潘佳奇采用了基于 Descartes 符号规则的分割区间的方法,在一个区间内只有一个实根的时候使用区间 Newton 迭代求根。

叶隽希采用了基于 Descartes 符号规则的二分分割区间的方法,在区间足够小时,使用 Newton 迭代求根。

在小组内的实现中,吕敬一的实现运行效率最高。潘佳奇的实现使用了区间 Newton 迭代法,相对普通 Newton 迭代法,采用区间运算可以保证最后不丢根且收敛。叶隽希的实现提供了一种十分简明的求根方法,其代码复杂度是三者最低的。

由于优化不够充分,小组实现的基于 Descartes 符号规则的方法未能发挥其理论上的效率。在小组的实现中有过度依赖高精度有理数等问题。经过测试使用高精度实数能带来更高的效率,但是会大量增加调试和测试的负担。采用精细实现的基于高精度实数的算法,是该项目一个可选的改进方向。

在这个项目中,各成员的主要贡献如下:
\begin{itemize}
	\item 吕敬一:文档编写、基础定理证明、项目代码整合、用户接口。
	\item 潘佳奇:文献收集、推导和编写公用函数、寻找优化方向、引入 gmp 库。
	\item 叶隽希:维护项目结构、小组研究方向、数据测试系统、管理外部依赖以及项目编译。
\end{itemize}

得益于三人合作,最终得到一个完整的项目。小组三人在这门课中收获颇丰。为此,感谢雍俊海老师的教导。
