% \documentclass{article}[UTF-8]
\documentclass{ctexart}
\usepackage{pendulum}
\allowdisplaybreaks[0]
\newcommand\var{\mathrm{var}}

\begin{document}
	
	\begin{titlepage}
    
    \centering
    \vspace*{\fill}
    
    \vspace*{0.5cm}
    
    \huge\bfseries
    一元五次及以上多项式方程的实根求解
    
	\vspace*{0.5cm}
	
	\large 吕敬一 (lv-jy22@mails.tsinghua.edu.cn) \\ 潘佳奇 (panjq22@mails.tsinghua.edu.cn) \\ 叶隽希 (yejx22@mails.tsinghua.edu.cn)
	
	\today
	
	\vspace*{\fill}
	
	\end{titlepage}

    \tableofcontents
    \clearpage

	\section{问题描述}
		给定一元5次以上多项式方程的系数,求解方程实根的存在情况(包括是否存在,是否有重根)以及具体数值,并计算求得 double 类型数值的绝对误差,使其尽量小。

	\section{前置约定}

	\subsection{定义}

首先给出接下来会用到的一些定义:

\begin{definition}[邻域]~

	\textbf{$x_0 \in \mathbb{R}$ 的邻域 $U(x_0, \delta) = (x_0 - \delta, x_0 + \delta)$} 
	
	\textbf{$x_0 \in \mathbb{R}$ 的去心邻域 $\mathring{U}(x_0, \delta) = (x_0 - \delta, x_0) \cup (x_0, x_0 + \delta)$}
	
\end{definition}

\begin{definition}[多项式的取模与整除]~

	对于多项式 $f(x), g(x)$ ,记 $f(x)$ 对 $g(x)$ 取模的结果为 $\textrm{rem}(f(x), g(x))$ ,则 $\textrm{rem}(f(x), g(x))$ 是满足如下条件的唯一多项式( $h(x)$ 也是多项式):
	$$
	\left\{
	\begin{aligned}
		& f(x) = g(x)h(x) + \textrm{rem}(f(x), g(x)) \\
		& \deg \textrm{rem}(f(x), g(x)) < \deg g(x)
	\end{aligned}
	\right.
	$$
	
	如果 $\textrm{rem}(f(x), g(x)) = 0$ ,则称\textbf{ $f(x)$ 被 $g(x)$ 整除}或\textbf{ $f(x)$ 包含因子 $g(x)$ }。
	
\end{definition}

\begin{definition}[多项式的最大公因式]~
	
	如果多项式 $f(x)$ 和 $g(x)$ 都被多项式 $h(x)$ 整除,则称\textbf{ $h(x)$ 是 $f(x)$ 和 $g(x)$ 的公因式}。在 $f(x)$ 和 $g(x)$ 的公因式中次数最高且最高次项系数为1的被称为\textbf{ $f(x)$ 和 $g(x)$ 的最大公因式},记为 $\gcd(f(x), g(x))$ 。两个多项式的最大公因式唯一且非零。
	
\end{definition}

\subsection{基本事实}

在研究过程中,我们注意到如下基本事实:

\begin{fact}[介值定理]~
	
	对于 $\mathbb{R}$ 上的连续函数 $f(x)$,若 $f(a)f(b) < 0(a < b)$,则 $(a, b)$ 中存在 $f(x)$ 的零点。
	
\end{fact}

\begin{fact}[代数基本定理的推论]~

	任一实系数 $n$ 次多项式至多有 $n$ 个实根(含重根)。

\end{fact}

\begin{fact}[带余项的泰勒公式]~
	
	设 $f(x)$ 在 $x_0$ 的某个邻域内存在 $n$ 阶导数,则当 $x \rightarrow x_0$ 时,
	$$
	\begin{aligned}
	f(x) = \sum_{k = 0}^{n} \frac{f^{(k)}(x_0)}{k!} (x - x_0)^k + o((x - x_0)^n) \\
	f(x) = \sum_{k = 0}^{n - 1} \frac{f^{(k)}(x_0)}{k!} (x - x_0)^k + \frac{f^{(n)}(\xi)}{n!} (x - x_0)^n
	\end{aligned}
	$$
	
	其中 $\xi$ 在 $x_0$ 和 $x$ 之间。
	
\end{fact}

\begin{fact}[最大公因式的基本性质]~

	以下 $f(x), g(x), h(x)$ 为多项式,$a (a \neq 0)$ 为 $f(x)$ 的最高次项系数,$k$ 为任意非零实数。
	$$
	\begin{aligned}
	\gcd(f(x), kg(x)) & = \gcd(f(x), g(x)) \\
	\gcd(f(x), g(x)) &= \gcd(f(x), g(x) + h(x)f(x)) \\
	\gcd(f(x), 0) & = \frac{f(x)}{a} \\
	\gcd(f(x)h(x), g(x)h(x)) = h(x) \gcd(f(x), g(x))
	\end{aligned}
	$$

\end{fact}

\begin{fact}[多项式的微分性质]~

	任意多项式都是 $\mathbb{R}$ 上的连续函数;
	
	任意多项式的导数还是多项式;

	若某多项式在 $x = x_0$ 处取极小值,则存在 $\delta > 0$ 使得 $\forall x \in (x_0 - \delta, x_0), f'(x) < 0$ 且 $\forall x \in (x_0, x_0 + \delta), f'(x) > 0$ 。极大值情况同理。
	
\end{fact}

\subsection{引理}

并给出了如下引理:

\begin{lemma}[多项式重根数与高阶导数值的关系]~
	
	若多项式 $f(x)$ 有实根 $x_0$ ,则 $f(x)$ 在 $x = x_0$ 处的重根数为 $r (r \le \deg f)$ 是 $f^{(i)}(x_0) = 0, \ (i \in \{1, 2, \cdots, r - 1\})$ 且 $f^{(r)}(x_0) \neq 0$ 的充分条件。
	
	\contribution{吕敬一}
	
\end{lemma}

下文朴素二分法与Sturm-Newton算法中gb\_getRepetition()函数均为该引理的直接应用,其原理不再赘述。

\begin{proof}[多项式重根数与高阶导数值的关系]~
	
	据假设,多项式 $f(x)$ 含有因子 $(x - x_0)^r$ ,不妨设 $f(x) = (x - x_0)^r g(x)$ ,其中 $g(x)$ 不含因子 $(x - x_0)$ 。考虑 $f(x)$ 的 $1, 2, \cdots, r$ 阶导数:
	$$
	\begin{aligned}
		f^{(1)}(x) & = r(x - x_0)^{r - 1}g(x) + (x - x_0)^rg^{(1)}(x) \\
		f^{(2)}(x) & = r(r - 1)(x - x_0)^{r - 2}g(x) + 2r(x - x_0)^{r - 1}g^{(1)}(x) + (x - x_0)^rg^{(2)}(x) \\
		f^{(3)}(x) & = r(r - 1)(r - 2)(x - x_0)^{r - 3}g(x) + 3r(r - 1)(x - x_0)^{r - 2}g^{(1)}(x) + 3r(x - x_0)^{r - 1}g^{(2)}(x) \\
		& + (x - x_0)^rg^{(3)}(x) \\
		& \vdots \\
		f^{(r - 1)}(x) & = \sum_{i = 0}^{r - 1} \binom{r - 1}{i} (x - x_0)^{r - i}g^{(r - 1 - i)}(x) \prod_{j = 0}^{i - 1} (r - j) \\
		f^{(r)}(x) & = \sum_{i = 0}^{r} \binom{r}{i} (x - x_0)^{r - i}g^{(r - i)}(x) \prod_{j = 0}^{i - 1} (r - j) \\
	\end{aligned}		
	$$
	
	观察发现 $f^{(1)}(x), f^{(2)}(x), \cdots, f^{(r - 1)}(x)$ 中每一项都被 $(x - x_0)$ 整除,从而 $f^{(1)}(x) = f^{(2)}(x) = f^{(r - 1)}(x_0) = 0$ 。而 $f^{(r)}(x)$ 中恰有一项 $(x - x_0)^0g(x)r!$ 不含因子 $(x - x_0)$ ,故 $f^{(r)}(x_0) \neq 0$ 。
	\qedhere
	
	\contribution{吕敬一}
	
\end{proof}

\begin{lemma}[无重根多项式与其导函数的公因子]~
	
	对于无重根多项式 $f(x)$,及其导函数 $f'(x)$,有其最大公因子为 $1$,即 $\gcd(f(x), f'(x)) = 1$

	\contribution{潘佳奇}
	
\end{lemma}

\begin{proof}[无重根多项式与其导函数的公因子]~

	对于 $f(x)$ 得其分解 $f(x) = (x - x_1) (x - x_2) \dots (x - x_n)$,则 $f(x)$ 只可能有 $(x - x_i)$ 形式的因子。

	对于 $f'(x)$,有
	$$
	f'(x) = \sum_{i=1}^{n} -x_i (x - x_1) \cdots (x - x_{i - 1}) (x - x_{i + 1}) \cdots (x - x_n).
	$$

	假设 $f'(x)$ 被 $(x - x_i)$ 整除,则有
	$$
	f'(x) \equiv -x_i(x - x_1) \cdots (x - x_{i - 1}) (x - x_{i + 1}) \cdots (x - x_n) \pmod{(x - x_i)}.
	$$

	但是因为 $f(x)$ 没有重根,$(x - x_1) \cdots (x - x_{i - 1}) (x - x_{i + 1}) \cdots (x - x_n)$ 不被 $(x - x_i)$ 整除,与 $f'(x)$ 被 $(x - x_i)$ 整除矛盾。

	所以 $\gcd(f(x), f'(x)) = 1$。
	\qedhere

	\contribution{潘佳奇}
	
\end{proof}


	\section{算法设计: 朴素二分法}
	
	\subsection{前提准备}

根据事实,我们自然地考虑以二分法为主体求解,配合一系列对区间的其他划分、调整操作来处理特殊情况。

\begin{problem}[对可能存在的实根作界的估计]~
	
	由于我们计划以二分法为主体来进行求解,需要先确定实根的上下界。
	
	\contribution{吕敬一}
	
\end{problem}

\begin{theorem}[多项式实根的界]~
	
	多项式方程 $a_nx^n+\dots+a_1x+a_0 = 0$ 如果存在实根,则实根必在区间 $[-bound, bound]$ 中,其中
	$$
	bound = 1 + \max_{i = 0}^{n - 1} \left|\frac{a_i}{a_n}\right|
	$$
	
	\contribution[摘自维基百科\footnotemark]{吕敬一}
\end{theorem}
\footnotetext{\url{https://en.wikipedia.org/wiki/Geometrical_properties_of_polynomial_roots}}

\begin{proof}[多项式实根的界]~
	
	设 $x_0 \in \mathbb{R}^*$ 是原多项式的一个实根,$|x_0| \le 1$ 的情形自不必证。下设 $|x_0| > 1$,则有
	$$
	a_nx_0^n + a_{n - 1}x_0^{n - 1} + \cdots + a_0 = 0
	$$
	
	即
	$$
	\begin{aligned}
	|a_nx_0^n| & = \left|\sum_{i = 0}^{n - 1} a_ix^i\right| \\
	& \le \sum_{i = 0}^{n - 1} |a_ix_0^i| \\
	& \le \max_{i = 0}^{n - 1} |a_i| \sum_{i = 0}^{n - 1}|x_0^i| \\
	& = \max_{i = 0}^{n - 1} |a_i| \dfrac{|x_0|^n - 1}{|x_0| - 1} \\
	& < \max_{i = 0}^{n - 1} |a_i| \dfrac{|x_0|^n}{|x_0| - 1}
	\end{aligned}
	$$
	
	从而
	$$
	|x_0| - 1 < \dfrac{\max_{i = 0}^{n - 1} |a_i|}{|a_n|}
	$$
	
	即
	$$
	|x_0| < 1 + \max_{i = 0}^{n - 1} \left|\frac{a_i}{a_n}\right|
	$$
	\qedhere
	
	\contribution{吕敬一}
	
\end{proof}

在求得实根的界之后,设计了如下算法来求解实根。

\subsection{算法流程}

\begin{algorithm}[朴素二分法]~
	
	\textbf{步骤1.} 将区间 $[-bound, bound]$ 作为初始区间插入求解队列 q 中。
	
	\textbf{步骤2.} 若已经找到的实根数与多项式次数相同,或队列 q 为空,则跳转到 \textbf{步骤9}。否则,取出队列 q 最前端的区间 $[l, r]$ ,若区间未空则重新执行 \textbf{步骤2},否则计算其端点的函数值 leftVal 与 rightVal ,并与0进行比较(此处考虑容差)。
	
	\textbf{步骤3.} 若左端点或右端点函数值为 0,则调用 \verb|gb_getRepetition()| 计算该点处的重根数并将其加入答案列表中,然后调用 \verb|gb_findNotZero()| 缩小区间,使得端点函数值不为 0.
	
	\textbf{步骤4.} 若区间长度过短(小于容差值)则取其中点 $m$ 代入函数判断是否是 0,若是则同样计算重根数并加入答案,而后跳转到 \textbf{步骤2}.
	
	\textbf{步骤5.} 若区间两端函数值异号,则执行\textbf{步骤6、7}。否则,执行 \textbf{步骤8}.
	
	\textbf{步骤6.} 区间两端函数值异号,则根据介值定理,区间内必有实根存在,尝试找到其中一个根 $res$ 。记当前正在处理的区间为 $I = [l, r]$,取其中点 $m$,若 $m$ 处函数值为0或区间过短(小于容差),则记 $res$ 为 $m$ 并进入下一步;否则,若 $l$ 点和 $m$ 点函数值异号,则将 $I$ 置为 $[l, m]$ 并重新执行这一步;否则,若点 $r$ 和点 $m$ 函数值异号(这必然成立),则将 $I$ 置为 $[m, r]$ 并重新执行这一步。
	
	\textbf{步骤7.} 计算点 $res$ 处的重根数并将其加入答案列表,然后调用 \verb|gb_findNotZero()| 找到 $res$ 前后最近的函数值不为 0 的实数 $r_0, l_0$,并将 $[l, r_0]$ 和 $[l_0, r]$ 插入到队列 q 末尾。而后重新回到步骤2.
	
	\textbf{步骤8.} 在当前迭代深度不过大的情况下,将当前处理的区间等分为8段,并依次加入到队列q末尾。而后重新回到 \textbf{步骤2}.
	
	\textbf{步骤9.} 对于已求出的实根,考虑到过程中的一系列误差可能性,对根进行进一步去重。将答案列表从小到大排序,记上一个保留的根为 $x$,$x$ 初始为最小的实根。从头遍历,若当前实根减去 $x$ 大于容差,则将 $x$ 更新为当前实根;否则将 $x$ 的重复数加上当前实根的重复数,并设法删除当前实根。
	
	\textbf{算法结束。} 此算法由吕敬一进行了代码实现。
	
	\contribution{吕敬一}
	
\end{algorithm}

\subsection{数值验证}

由于本算法存在明显的正确性问题,暂不对其进行理论证明。关于数值验证涉及的具体方法,见本文第五节:数值验证的方法。

对算法1进行了数值验证,注意到三条主要问题:

(1) 偶数次重根会被直接忽略,或者在某些情况下在实际根附近求出若干不精确的根。初步判断为二分法的结构性缺陷(参考引理2),需要后续改进算法。

\begin{example}[]~
	
	当多项式为
	$$
	\begin{aligned}
	-0.0000000201268990957616007532531384642895955217056780384154990315 & + \\ 0.0000021236101838969952624312006322915280520646774675697088241577x & + \\ -0.0000503647104449539523523919626324385490079293958842754364013672x^2 & + \\ -0.000570193171519725052441562862526325261569581925868988037109375x^3 & + \\ 0.0156122311833811790171555600181818590499460697174072265625x^4 & + \\ -0.06494494310663577463227369435116997919976711273193359375x^5 & + \\ -0.1290991197533018775001067979246727190911769866943359375x^6 & + \\ 0.71293559156111252494980590199702419340610504150390625x^7 & + \\ 0.96451662622768574717468936796649359166622161865234375x^8 & + \\ -2.00802000715429596056083028088323771953582763671875x^9 & + \\ -3.3939905177597129437572220922447741031646728515625x^{10} & + \\ 0.3866014721655073316242123837582767009735107421875x^{11} & + \\ 2.5719003370867863367266181739978492259979248046875x^{12} & + \\ 
	x^{13}
	\end{aligned}
	$$ 
	时(精确实根为 -1.43822, -0.910655, -0.844453, -0.743001, -0.600247, -0.0570855, 0.0187824, 0.0187824, 0.106847, 0.197466, 0.197466, 0.490085, 0.992328 ),所求出的根为 -1.43822, -0.910654, -0.844448, -0.743005, -0.600246, -0.0570885, 0.106844, 0.49008, 0.992327,不难发现各出现了两次的 0.0187824 和 0.197466 被漏掉了。

	当多项式为
	$$
	(x + 0.88971571859204046095470630461932159960269927978515625)^{10}
	$$
	时(精确实根为 -0.88971571859204046095470630461932159960269927978515625,重根10次),所求出的实根为-0.930381和-0.849622,结果精度非常低。 
	
	\contribution{吕敬一}

\end{example}

(2) 某些根与实际根的误差超过了容差。有可能是代码实现问题,也有可能是多项式求值时容差选取不够优秀,还有可能是浮点数自身的误差导致。

\begin{example}[]~
	
	当多项式为
	$$
	\begin{aligned}
	-0.0000058064403473063785275599599233764536165836034342646598815918 & + \\ 0.0002874957679957463956531282800455073811463080346584320068359375x & + \\ -0.00613773044503206598176969777114209136925637722015380859375x^2 & + \\ 0.07348729113091197195917203544013318605720996856689453125x^3 & + \\ -0.53638136681462744714821155866957269608974456787109375x^4 & + \\ 2.420421284825134744522756591322831809520721435546875x^5 & + \\ -6.490042806599969793523996486328542232513427734375x^6 & + \\ 9.2100965283636497815678012557327747344970703125x^7 & + \\ -5.249481342694291896577851730398833751678466796875x^8 & + \\ 
	x^9
	\end{aligned}	
	$$
	时(精确实根为7次重根的0.144122和两次重根的2.12031),求出的实根为0.145712,其与0.144122的误差超过了预设的容差 $10 ^ {-5}$。
	
	\contribution{吕敬一}
	
\end{example}

(3) 在算法1的步骤8中进行了将区间等分并递归求解的操作。这一操作是为了在两端函数值同号的区间中找到两端函数值异号的区间以进行二分,但实际上产生了很多不必要的区间,造成了巨大的时间开销,程序运行缓慢。

	
	\section{算法设计:Sturm-Newton算法}
	
	从上述问题中可以看出,找到有根区间是一个较大的难点。此外,可能需要考虑二分法以外的数值方法来应对偶数次重根的情况。
	
	\subsection{前提准备}

\begin{problem}[实根分离]~
	
	对于没有重根的多项式,实根分离是构造一系列不相交的区间,使得这些区间不相交,且每个区间内有且只有一个实根,并且多项式所有的实根都被区间覆盖到。
	
	\contribution[摘自维基百科\footnotemark]{叶隽希}
	
\end{problem}
\footnotetext{\url{https://en.wikipedia.org/wiki/Real-root\_isolation}}

使用实根分离可以更好地使用迭代法,因为每个区间只有一个实根,那么迭代法就能快速地找到多项式所有的实根。我们首先提出了用 Sturm 定理来解决实根分离问题的算法。定理说明如下:

\begin{definition}[Sturm 序列]~
	
	对于多项式 $f$ ,定义 $f$ 的\textbf{Sturm 序列} $T$ 如下:
	$$
	\begin{aligned}
		& T_0 = f \\
		& T_1 = f' \\
		& T_{n + 1} = -\textrm{rem}(T_{n - 1}, T_n) \ \ \ \ (n \ge 1, \ \deg {T_n} \ge 1)
	\end{aligned}
	$$
	
	其中 $\textrm{rem}(f, g)$ 是唯一满足 $f = gh + \textrm{rem}(f, g)$,且 $\deg \textrm{rem}(f, g) \le \deg g - 1$ 的多项式。此处 $h$ 亦为某多项式。
	
	\contribution[摘自维基百科]{吕敬一}
	
\end{definition}

\begin{definition}[实数在多项式序列中的变号数]~
	
	对于多项式序列 $T_0, T_1, \cdots, T_m$ 和 $x \in \mathbb{R}$,记数列 $\{v_n\}$ 为 $v_i = T_i(x) \ (0 \le i \le m)$。
	
	定义\textbf{ $x$ 在 $T$ 中的变号数} 为 $\{v_n\}$ 中满足 $v_i \neq 0$ 且 $v_i$ 与其前一个非零项异号的正下标 $i$ 的数量,记作 $\var(T, x)$ 。
	
	\contribution[摘自维基百科]{吕敬一}
	
\end{definition}

\begin{theorem}[Sturm 定理]~
	
	若 $a, b \ (a < b)$ 都不是多项式 $f$ 的零点,则 $f$ 在 $(a, b)$ 上的不同零点个数为 $\var(T, a) - \var(T, b)$ ,其中 $T$ 为 $f$ 的 Sturm 序列。 
	
	\contribution[摘自维基百科]{吕敬一}
	
\end{theorem}

根据定理2,利用上一节中子问题的结果,结合牛顿迭代法(步骤6、7、8)设计了如下算法:

\subsection{算法流程}

\begin{algorithm}[Sturm-Newton 算法]~
	
	\textbf{步骤1.} 对于输入的多项式 $f$ ,求出其 Sturm 序列 $T$ 备用。
	
	\textbf{步骤2.} 将区间 $[-bound, bound]$ 作为初始区间插入求解队列q中。
	
	\textbf{步骤3.} 若q为空,则跳转至步骤9;否则,取出队列q最前端的区间 $[l, r]$。若 $f(l)$ 在容差允许范围内等于0,则将 $l$ 加入答案列表,并调用 \verb|gb_findNotZero()| 找到大于 $l$ 的最近的使得 $f(l')$ 在容差范围内不为0的浮点数 $l'$,并令 $l := l'$ 。对于 $r$ 同理亦然。
	
	\textbf{步骤4.} 利用 Sturm 定理计算 $[l, r]$ 中不同实根的数量 rootNum。若 rootNum 大于等于2,或 rootNum 等于1且 $r - l$ 大于等于给定阈值 $L$,则进入步骤5;若 rootNum 等于0,跳转至步骤3;否则,跳转至步骤6。
	
	\textbf{步骤5.} 若 rootNum 大于等于2,则将 $[l, r]$ 作二等分;否则将 $[l, r]$ 作 $k$ 等分。把等分所得区间插入队列q末尾,回到步骤2。
	
	\textbf{步骤6.} 令 $x_0 := \frac{l + r}{2}$ 。不断执行步骤7,直至步骤7已被执行 ITER\_LIM 次或 $|\frac{f(x_0)}{f'(x_0)}| < \frac{e}{10}$ 。
	
	\textbf{步骤7.} 若 $f'(x_0) \neq 0$,则令 $x := x_0 - \frac{f(x_0)}{f'(x_0)}$;否则,调用 \verb|gb_findNotZero()| 找到大于 $x_0$ 的最近的使得 $f'(x_1) \neq 0$ 的浮点数 $x_1$ ,并令 $x := x_1 - \frac{f(x_1)}{f'(x_1)}$ 。最后,再置 $x_0 := x$ 。
	
	\textbf{步骤8.} 将 $x_0$ 加入答案列表,跳转至步骤3。
	
	\textbf{步骤9.} 对于答案列表中的每一个根 $x$,调用 \verb|gb_getRepetition()| 计算其重根数。若 $f'(x) \neq 0$,将 $|\frac{f(x)}{f'(x)}|$ 记为其误差;否则,将其误差记为0。
	
	\textbf{步骤10.} 按照要求,根据答案列表中根的情况设置变量 flag 的值。
	
	\textbf{算法结束。} 此算法由吕敬一进行了代码实现。
	
	\contribution{吕敬一}
	
\end{algorithm}

在实际操作中,取 $L = 0.1, k = 2, \textrm{ITER\_LIM} = 100$ 。关于 $k$ 的取值,在算法设计时进行了分析,详见下一小节理论证明。

\subsection{理论证明}

	\begin{theorem}[算法2中区间等分数 k 的最优值为2]~
		
		证明:
		
		对于一个有一个根 $x_0$ 但未作求解的区间 $I$,设其长度为 $l \ (l > L)$,设单次求解某实数在 Sturm 序列中变号数的时间开销为 $s$,每个区间 $k$ 等分需要计算 $k - 1$ 个分割点的变号数(区间端点由于上一层分割时已计算过,可用 unordered\_map 存储下来,不再计算)。设由 $I$ 最终求出长度小于 $L$ 的只有一个根的区间的时间开销为 $f(k, l)$,
		$$
		\begin{aligned}
			f(k, l) & = (k - 1)s \log_k \frac{l}{L} \\
			& = s\ln \frac{l}{L} \cdot \frac{k - 1}{\ln k} 
		\end{aligned}
		$$
		
		记 $g(x) = \frac{x - 1}{\ln x} \ (x \ge 2)$, 有 $g'(x) = \frac{\ln x - 1 + \frac{1}{x}}{(\ln x) ^ 2}$,
		注意到 $x \in [3, +\infty)$ 时,$\ln x - 1 + \frac{1}{x} > 0$ ,则 $g(x)$ 在 $[3, +\infty)$ 的最小值为 $g(3) = \ln 3 - \frac{2}{3} > \ln 2 - \frac{1}{2}$ 。而算法流程中 $k$ 为大于等于2的整数,故 $k = 2$ 时对于任意给定的 $l$ ,$f(k, l)$ 取得最小值。
		\qedhere
		
		\contribution{吕敬一}
		
	\end{theorem}

	现补充 Sturm 定理以及步骤6、7、8所涉及之牛顿迭代法的证明。
	
	\begin{proof}[Sturm 定理]~
		
		对于多项式 $f(x)$ 和 $a, b \in \mathbb{R} \ (a < b)$ ,记 $f(x)$ 的 Sturm 序列为 $T: \ T_0, T_1, \cdots, T_m$ 。假设 $a, b$ 均不是 $f(x)$ 的根。
		
		首先证明 $T$ 的性质:任意 $1\le i < m, \gcd(T_i, T_{i + 1}) = \gcd(T_0, T_1)$ 。
			
		\subsubsection*{\ \ \ \ \ \ Sturm 序列的性质}
			
			对于 $i = 1$ 有
			$$
			\begin{aligned}
			\gcd(T_1, T_2) & = \gcd(T_1, -\textrm{rem}(T_0, T_1)) \\
						   & = \gcd(T_1, \textrm{rem}(T_0, T_1)) \\
						   & = \gcd(T_1, T_0 - h(x)T_1) \\
						   & = \gcd(T_0, T_1) 
			\end{aligned}
			$$
			
			其中 $h(x)$ 是某多项式。同理依次可证明 $i = 2, 3, \cdots, m - 1$ 的情形。
		
		不妨设 $d(x) = \gcd(f(x), f'(x))$ 。考虑将 $f(x)$ 的实根对应的因式分离出来,则 $f(x)$ 可表示为
		$$
		f(x) = g(x)\prod_{i = 1}^k (x - x_i)^{r_i}
		$$
		
		其中 $x_1, x_2, \cdots, x_k$ 为 $f(x)$ 的全体实根,$r_1, r_2, \cdots, r_k$ 为对应的重根数。注意到
		$$
		f'(x) = \sum_{i = 1}^k \frac{r_i}{x - x_i} g(x)\prod_{j = 1}^k (x - x_j)^{r_j} + g'(x)\prod_{j = 1}^k (x - x_j)^{r_j}
		$$
		
		则使得 $f'(x)$ 能被 $(x - x_i)^t$ 整除的最高次数 $t = r_i - 1$ ,进而 
		$$
		d(x) = \gcd(f(x), f'(x)) = h(x)\prod_{i = 1}^k (x - x_i)^{r_i - 1}
		$$
		
		其中 $h(x)$ 是不被任意 $(x - x_i)$ 整除的某多项式。由 $f(x)$ 被 $h(x)$ 整除易知 $h(x)$ 无实根。
		
		定义多项式序列 $S: \ S_i = \frac{T_i}{d(x)}, \ 0 \le i \le m$ ,则 $\forall 0 \le i < m, \gcd(S_i, S_{i + 1}) = \dfrac{\gcd(T_i, T_{i + 1})}{d(x)} = \dfrac{\gcd(T_0, T_1)}{d(x)} = 1$ 。这即说明 $\forall x \in \mathbb{R}$ ,$S_i(x)$ 和 $S_{i + 1}(x)$ 不同时为0。\textbf{换言之,$S$ 中任意相邻两项不会同时取值为0。}
		
		此外,考虑 $S$ 的首项,有
		$$
		S_0 = \dfrac{T_0}{d(x)} = \frac{g(x)}{h(x)}\prod_{i = 1}^k (x - x_i) 
		$$
		
		注意到 $g(x)$ 无实根,因而 $S_0$ 的全体不同实根为 $x_1, x_2, \cdots, x_k$ ,\textbf{与 $f(x)$ 相同。值得注意的是,$S_0$ 没有重根。}
		
		由于 $a$ 不是 $f(x)$ 的根,$a - x_i \neq 0$ 对 $1 \le i \le k$ 成立。又由 $h(x)$ 无实根知 $d(a) \neq 0$ ,从而 $T_i(a)$ 与 $S_i(a)$ 要么均为0,要么相差一个非零常数倍,\textbf{因而有 $\var(T, a) = \var(S, a)$ 。同理 $\var(T, b) = \var(S, b)$ 。}
		
		\textbf{故只要证 $S_0$ 在 $(a, b)$ 上的实根数为 $\var(T, a) - \var(T, b)$ 即可。} 考虑变量 $x$ 从 $a$ 连续变化至 $b$ 的过程中 $\var(S, x)$ 的变化,不难发现仅当 $x$ 经过某个 $S_i$ 的根时 $\var(S, x)$ 可能发生变化。作如下讨论:
		
		\subsubsection*{\ \ \ \ \ \ $x$ 经过 $S_0$ 的根}
			
			设 $x$ 经过了 $S_0$ 的根 $x_0$ ,有 $S_0(x_0) = f(x_0) = 0$ ,进而根据事实2必存在 $x_0$ 的去心邻域 $\mathring{U}(x_0, \delta)$ 使得该邻域内 $f(x_0)$ 非零。
			
			注意到 $f(x)f'(x) = T_0T_1 = S_0S_1(d(x))^2$ ,知 $f(x)f'(x)$ 与 $S_0S_1$ 同号。又设 $p(x) = \frac{1}{2}(f(x))^2$ ,有 $p'(x) = f(x)f'(x)$ 且 $p(x)$ 在 $\mathring{U}(x_0, \delta)$ 恒正。
			
			故根据事实5知存在 $\epsilon \in (0, \delta] $ 使得 $\forall x \in (x_0 - \epsilon, x_0), p'(x) = f(x)f'(x) < 0$ 且 $\forall x \in (x_0, x_0 + \epsilon), f(x)f'(x) > 0$ 。
			
			换言之,在 $x$ 经过 $x_0$ 时,$S_0(x)S_1(x)$ 从负变为正			,$S_0(x)$ 与 $S_1(x)$ 从异号变为同号,$\var(S, x)$ 减少1。
		 
		\subsubsection*{\ \ \ \ \ \ $x$ 经过 $S_i (1 \le i \le m)$ 的根}
		
			若 $i = m$ 且 $S_m = 0$ 则对 $\var(S, x)$ 没有影响,不作考虑。
			
			否则,设 $x$ 经过了 $S_i$ 的根 $x_0$ ,有 $S_i(x_0) = 0, S_{i + 1}(x_0) \neq 0, S_{i - 1}(x_0) \neq 0$ 。根据 Sturm 序列的定义知存在多项式 $q(x)$ 满足
			$$
			T_{i - 1} = T_iq(x) - T_{i + 1}
			$$
			
			即
			$$
			S_{i - 1}d(x) = S_id(x)q(x) - S_{i + 1}d(x)
			$$
			
			由于 $d(x) = \gcd(f(x), f'(x))$ 非零,有
			$$
			\begin{aligned}
			& S_{i - 1} = S_iq(x)- S_{i + 1} \\
			& S_{i - 1}(x_0) = S_i(x_0)q(x_0) - S_{i + 1}(x_0) = -S_{i + 1}(x_0) 
			\end{aligned}
			$$
			
			这即说明 $S_{i - 1}(x_0), S_{i + 1}(x_0)$ 异号,进而存在 $x_0$ 的某个邻域 $U(x_0, \delta)$ 使得 $S_{i - 1}, S_{i + 1}$ 在该邻域不变号且彼此异号,同时 $S_i$ 在该邻域非零。
			
			不难发现当 $x \in \mathring{U}(x_0, \delta)$ 时,无论 $S_i(x)$ 符号如何,$S_{i - 1}(x)$ 和 $S_{i + 1}(x)$ 中必恰有一者与之同号,一者与之异号,故在该邻域内 $(S_{i - 1}, S_i, S_{i + 1})$ 对 $\var(S, x)$ 的贡献总为1,$\var(S, x)$ 不变。
			
		综上,$S_0$ 在 $(a, b)$ 的实根数恰等于 $\var(S, a) - \var(S, b)$ ,即 $f(x)$ 在 $(a, b)$ 的实根数恰等于 $\var(T, a) - \var(T, b)$ 。
		\qedhere
		
		\contribution{吕敬一}
		
	\end{proof}

	\begin{proof}[牛顿迭代法的收敛性]~
		
		设五次及以上多项式 $f(x)$ 有实根 $\alpha$ 。若 $\alpha$ 处有重根, 由引理2,$f'(\alpha) = 0$ 。注意到 $f'(x)$ 也是多项式,由事实2知存在 $\alpha$ 的去心邻域 $\mathring{U}(\alpha, \delta)$ 满足 $\forall x \in U(\alpha, \delta), \ f'(x) \neq 0$ 。若 $\alpha$ 处无重根,由引理2,$f'(\alpha) \neq 0$ ,进而由连续性可知存在邻域 $U(\alpha, \delta)$ 使得 $\inf\limits_{x \in U(\alpha, \delta)} |f'(x)| > 0$。
		
		\textbf{下面的讨论在 $U(\alpha, \delta)$ 或 $\mathring{U}(\alpha, \delta)$ 中进行,并假设 $\delta$ 足够小。}设牛顿迭代的一系列猜测值为 $x_0, x_1, \cdots$ ,满足 $x_n \in U(\alpha, \delta)$ 。记 $d_n = |x_n - \alpha|$ 。
		\subsubsection*{\ \ \ \ \ \ $\alpha$ 处无重根}
		根据步骤7,运用事实3,有
		$$	
		f(x_n) + f'(x_0)(\alpha - x_n) + \frac{f''(\xi)}{2} (\alpha - x_n)^2 = f(\alpha) = 0
		$$
		
		则
		$$
		\begin{aligned}
		d_{n + 1} & = \left|x_{n + 1} - \alpha \right| \\
		& = \left|x_n - \frac{f(x_n)}{f'(x_n)} - \alpha\right| \\
		& = \left|x_n - \alpha - \frac{f(x_n)(x_n - \alpha) - \frac{f''(\xi)}{2}(x_n - \alpha)^2}{f'(x_n)} \right| \\
		& = \left|-\frac{f''(\xi)(x_n - \alpha)^2}{2f'(x_n)}\right| \\
		& = d_n^2\left|\frac{f''(\xi)}{2f'(x_n)}\right| \\
		\end{aligned}
		$$
		
		由于 $x_n \in U(\alpha, \delta)$ ,$|f'(x_n)| \ge \inf\limits_{x \in U(\alpha, \delta)} |f'(x)| > 0$ ,故 $\frac{f''(\xi)}{2f'(x_n)}$ 有界。
		
		不妨设 $\left|\frac{f''(\xi)}{2f'(x_n)}\right| \le M$ ,立得:
		
		\begin{equation}
			d_{n + 1} \le Md_n^2
		\end{equation}
		
		而 $d_0 = |x_0 - \alpha| < \delta$ 很小,可以认为 $Md_0 < 1$ 。故有 $d_1 \le Md_0^2 < d_0$ 。
		假设 $d_n < d_{n - 1} < \cdots < d_1 < d_0$ ,则
		$$
		d_{n + 1} \le Md_n^2 < Md_0d_n < d_n
		$$
		
		由第二数学归纳法知 ${d_n}$ 单调递减,进而 $\{d_n\}$ 收敛。
		
		不妨设 $\lim\limits_{n \rightarrow \infty} d_n = A (0 \le A < \frac{1}{M})$ ,对 $(1)$ 式两侧取极限,得
		$$
		A \le MA^2
		$$
		
		解得 $A \in (-\infty, 0] \cup [\frac{1}{M}, +\infty)$ 。结合 $0 \le A < \frac{1}{M}$ 知 $A = 0$ 。
		
		故牛顿迭代法在 $\alpha$ 附近二阶收敛。
		
		\subsubsection*{\ \ \ \ \ \ $\alpha$ 处有重根}
		由引理2,设重根数为 $m$ ,则 $f^{m}(x) \neq 0$ 。根据步骤7和事实3有:
		$$
		\begin{aligned}
		d_{n + 1} & = \left|x_{n + 1} - \alpha \right| \\
		& = \left|x_n - \frac{f(x_n)}{f'(x_n)} - \alpha\right| \\
		& = \left|x_n - \alpha - \dfrac{\dfrac{f^{(m)}(\alpha)(x_n - \alpha)^m}{m!} + o((x_n - \alpha)^m)}{\dfrac{f^{(m)}(\alpha)(x - \alpha)^{m - 1}}{(m - 1)!} + o((x_n - \alpha)^{m - 1})} \right| \\
		& = \left|\dfrac{\dfrac{(m - 1)f^{(m)}(\alpha)(x_n - \alpha)^m}{m!} + o((x_n - \alpha)^m)}{\dfrac{f^{(m)}(\alpha)(x - \alpha)^{m - 1}}{(m - 1)!} + o((x_n - \alpha)^{m - 1})}\right| \\
		& = \frac{m - 1}{m}d_n
		\end{aligned}
		$$
		
		故牛顿迭代法在 $\alpha$ 附近线性收敛。
		\qedhere
		
		\contribution[部分参考自维基百科,大部分独立提出]{吕敬一}
		
	\end{proof}

	这说明了算法2中步骤6、7、8的有效性:只要实根分离出的区间足够小,便能收敛至区间之内的实根。关于所需区间长度的理论分析,我们没有得到理想的结果,只得到了如下对绝大部分情况适用但实际操作中不易实施的结论:
	
	\begin{theorem}[无重根条件下理想收敛结果的初值条件]~
		
		\textbf{下面假设多项式 $f(x)$ 没有重根。}
		
		假设已知一区间 $[l, r]$ 使得存在恰好一个 $\alpha \in [l, r]$ 是 $f(x)$ 的实根。根据定理2,$f’(\alpha) \neq 0$ 。\textbf{假设}存在 $\alpha$ 的一个邻域 $U(\alpha, \delta)$ 使得邻域内 $f'(x), f''(x), f'''(x)$ 均不变号,即邻域内 $f(x), f'(x), f''(x)$ 单调。
		
		当满足
		$$
		\left\{
		\begin{aligned}
			& r - l < \dfrac{4\min\{|f'(l)|, |f'(r)|\}}{\max\{|f''(l)|, |f''(r)|\}} \\
			& [l, r] \subset U(\alpha, \delta)
		\end{aligned}
		\right.
		$$
		
		时,选取初值 $\frac{l + r}{2}$ ,牛顿迭代将收敛至 $\alpha$ 。
		
		\contribution{吕敬一}
		
	\end{theorem}

	\begin{proof}[]~
		
		我们只要证明 $Md_0 < 1$ 即可($M$ 的定义见证明5)。
		
		由于 $\alpha \in [l, r]$ ,有 $d_0 = \left|\frac{l + r}{2} - \alpha\right| \le \frac{r - l}{2}$ 。
		
		而根据 $f'(x), f''(x)$ 的单调性,不难得到
		$$
		\left|\frac{f''(\xi)}{2f'(x_n)}\right| \le \dfrac{\max\{|f''(l)|, |f''(r)|\}}{2\min\{|f'(l)|, |f'(r)|\}}
		$$
		
		令 $M = \dfrac{\max\{|f''(l)|, |f''(r)|\}}{2\min\{|f'(l)|, |f'(r)|\}}$ ,立得
		$$
		\begin{aligned}
		Md_0 & = \dfrac{\max\{|f''(l)|, |f''(r)|\}}{2\min\{|f'(l)|, |f'(r)|\}} \cdot \frac{r - l}{2} \\
		& < \dfrac{\max\{|f''(l)|, |f''(r)|\}}{2\min\{|f'(l)|, |f'(r)|\}} \cdot \dfrac{2\min\{|f'(l)|, |f'(r)|\}}{\max\{|f''(l)|, |f''(r)|\}} \\
		& = 1
		\end{aligned}
		$$
		\qedhere
		
		\contribution{吕敬一}
		
	\end{proof}

	由于假设条件不一定能满足且难以验证,这个结论缺乏实用性,我们在实际算法设计中选择了手动估计区间长度阈值 $L$ 。
	
	另外,算法2的步骤6和步骤9都涉及了根据当前点的多项式值和导数值估计当前的误差。现形式化地补充该定理及其证明:
	
	\begin{theorem}[微小误差估计]~
		
		设 $x_0$ 是多项式 $f(x)$ 的一个实根,$x - x_0$ 充分小且 $f'(x) \neq 0$ ,则
		$$
		x - x_0 \approx \frac{f(x)}{f'(x)}
		$$
		
		\contribution{吕敬一}
	
	\end{theorem}

	\begin{proof}[微小误差估计]~
	
		运用事实3,在 $x$ 处对 $f(x)$ 泰勒展开,得
		$$
		f(x_0) = f(x) + f'(x)(x_0 - x) + o(x_0 - x)
		$$
		
		注意到 $x_0$ 是 $f(x)$ 的根,$f(x_0) = 0$ ,即有
		$$
		x - x_0 = \frac{f(x)}{f'(x)} + o(1)
		$$
		
		由 $x - x_0$ 极小,忽略 $o(1)$ 项即证。
		\qedhere
		
		\contribution{吕敬一}
		
	\end{proof}
\subsection{数值验证}

运用7.1节的方法对算法2进行了数值验证,注意到三条主要问题:

(1) 在利用 Sturm 定理计算某些区间所含不同实根数时,有时会出现错误,导致最后出现丢根的情况。极端情况下甚至可能将有实根的多项式方程判定为没有实根。经过分析,这种情况几乎只出现在包含重根的区间上。

\begin{example}[]~
	
	当多项式为例子1中的第一个13次多项式时,算法2得到的实根为
	$$
	\begin{aligned}
	& -1.4382158100094859509709976919111795723438262939453125 \\ & -0.91065489381036457405826922695268876850605010986328125 \\  & -0.844452842338522913223641808144748210906982421875 \\ & -0.743000962121088459610973586677573621273040771484375 \\ & -0.60024708012588556815813944922410883009433746337890625 \\ & -0.057085484496008483124018795251686242409050464630126953125 \\ & 0.018782197988160963253445601139901555143296718597412109375 \\ & 0.018782197988160963253445601139901555143296718597412109375 \\ & 0.106847165384948372679474459800985641777515411376953125 \\ & 0.490084625174973254058130578414420597255229949951171875 \\ & 0.9923277860580725917571953687001951038837432861328125 \\
	\end{aligned}
	$$
	
	与例子1中的精确实根对比,两次重根的0.197466被遗漏了。
	
	当多项式为
	$$
	(x - 1.0782053783126188672980561022995971143245697021484375)^8
	$$
	时,显然有实根1.0782053783126188672980561022995971143245697021484375,但算法2返回结果为不存在实根。
	
	\contribution{吕敬一}
	
\end{example}

(2) 对包含一个重根次数比较多的实根的区间进行牛顿迭代法求解时,无论如何增加迭代次数,根的精度都非常低,无法达到要求。

\begin{example}[]~
	
	当多项式为例子2中的多项式时,算法2求出的实根为0.1458024118246819544886250241688685
	49168109893798828125,与0.144122的误差很大,超过了预设的容差 $10^{-6}$ 。
	
	\contribution{吕敬一}
	
\end{example}

(3) 在步骤9中计算重根数时,有时计算结果并不准确。初步分析是因为没有控制好高阶导数值的容差,或是不可避免的浮点运算精度丢失导致。

\begin{example}[]~
	
	当多项式为
	$$
	(x - 0.426447228330617933433899224837659858167171478271484375)^2
	$$
	时(精确实根为 0.426447228330617933433899224837659858167171478271484375,重根两次),算法2返回了0.426447224569169336394480751550872810184955596923828125,但给出重根数为1。
	
	\contribution{吕敬一}
	
\end{example}

	
	\section{算法设计:基于笛卡尔符号规则的实根分离}
	
	\subsection{前提准备}

\subsubsection{重根的处理}

	根据前述算法的测试结果,多项式的重根会严重影响实根分离算法的表现。为此,提出了如下子问题:

\begin{problem}[平方因子分解]~

	对于有重根的多项式,其要转化为没有重根的多项式以方便计算。

    亦即,要找到一个新多项式,满足两个多项式零点相同,但新多项式方程没有重根。

	\contribution{潘佳奇}

\end{problem}

经过检索与学习,提出了如下算法解决该子问题:

\begin{algorithm}[Yun's algorithm]~

	对于要分解的非零多项式,假设最终 $k$ 重根 $(x - a)^k$ 被分解到多项式 $a_k(x)$ 里,设
	$$
	f(x) = a_1(x) a_2^2(x) a_3^3(x) \cdots a_k^k (x).
	$$

	则有初始化:
	$$
	\begin{aligned}
		b_0(x) & := f(x); \\
		d_0(x) & := f'(x); \\
		i & := 0. \\
	\end{aligned}
	$$

	以及迭代过程:重复以下过程直到 $b_i = 1$。
	$$
	\begin{aligned}
		a_i(x) & := \gcd(b_i(x), d_i(x))); \\
		i & := i + 1; \\
		b_{i}(x) & := \frac{b_{i-1}(x)}{a_{i-1}(x)}; \\
		c_{i}(x) & := \frac{d_{i-1}(x)}{a_{i-1}(x)}; \\
		d_{i}(x) & := c_{i}(x) - b'_{i}(x). \\
	\end{aligned}
	$$

	计算出的 $a_i(x)$ 即为要求的结果。

	\contribution[摘自维基百科 \footnotemark]{潘佳奇}
	
\end{algorithm}
\footnotetext{\url{https://en.wikipedia.org/wiki/Square-free\_polynomial}}

\begin{proof}[Yun's algorithm]~

	对于 $f$ 的分解:
	$$
	f(x) = a_1(x) a_2^2(x) a_3^3(x) \cdots a_k^k (x).
	$$

	令 $a_0(x) = a_2^1(x) a_3^2(x) \cdots a_k^{k - 1}(x)$,则对于 $f(x)$,有
	$$
	f(x) = a_0(x) a_1(x) \cdots a_k(x)
	$$

	对于 $f'(x)$,有
	$$
	f'(x) = a_0(x) \sum_{i=1}^{k} i a_i'(x) a_1(x) \cdots a_{i - 1}(x) a_{i + 1}(x) \cdots a_k(x)
	$$

	与前置约定中的『无重根多项式与其导函数最大公因子为 $1$』同理,因为 $a_1(x) a_2(x) \cdots a_k(x)$ 没有重根,所以有 $gcd(\frac{f(x)}{a_0(x)}, \frac{f'(x)}{a_0(x)}) = 1$。

	所以 $\gcd(f(x), f'(x)) = a_0(x)$。

	于是令 $b_1(x) = \frac{f(x)}{a_0(x)}, c_1(x) = \frac{f'(x)}{a_0(x)}, d_1(x) = c_1(x) - b_1'(x)$,则
	$$
	d_1(x) = a_1(x) \sum_{i=2}^{k} (i - 1) a_i'(x) a_2(x) \cdots a_{i - 1}(x) a_{i + 1}(x) \cdots a_k(x)
	$$

	再次引用『多项式与其导函数公因子』的引理,于是有 $\gcd(d_1(x), b_1(x)) = a_1(x)$

	于是由迭代过程,有
	$$
	b_2(x) = a_2(x) \cdots a_k(x),
	$$
	$$
	c_2(x) = \sum_{i=2}^{k} (i - 1) a_i'(x) a_2(x) \cdots a_{i - 1}(x) a_{i + 1}(x) \cdots a_k(x).
	$$

	由归纳可证,对于任意 $\xi \in \mathbb{N}^+, \xi \leq k$,有
	$$
	a_\xi = \gcd(b_{\xi}, d_{\xi})
	$$
	$$
	b_\xi(x) = a_\xi(x) \cdots a_k(x),
	$$
	$$
	c_\xi(x) = \sum_{i=\xi}^{k} (i - \xi + 1) a_i'(x) a_\xi(x) \cdots a_{i - 1}(x) a_{i + 1}(x) \cdots a_k(x).
	$$

	即迭代过程每步提取出了正确的 $a_i (i \geq 1)$。\qedhere

	\contribution[证明]{潘佳奇}

\end{proof}

\subsubsection{实根分离}

解决重根问题之后,再给出如下定义与定理,以实现更高效的实根分离:

\begin{definition}[实根数]~

	对于多项式 $f(x) = a_n x^n + \dots + a_1 x + a_0$,定义实根数 $\varrho(f(x))$ 为 $f(x)$ 的实根数(重根计算多次,下同),$\varrho_+(f(x))$ 为 $f$ 的正实根数,$\varrho_-(f(x))$ 为 $f$ 的负实根数。$\varrho_{l,r}(f(x))$ 为区间 $(l, r]$ 之间 $f$ 的实根数。

	\contribution{潘佳奇}
	
\end{definition}

\begin{definition}[多项式的变号数]~

	对于多项式 $f(x) = a_n x^n + \dots + a_1 x + a_0$,定义变号数 $\var(f)$ 为序列 $[a_n, a_{n-1}, \dots, a_0]$ 的变号次数,即忽略 $0$ 的情况下,正负号变了多少次。

	形式化地讲,$\var(f)$ 就是有多少对二元组 $(i, j)$ 满足 $i < j, a_i a_j < 0$,且 $\forall i < k < j, a_k = 0$。

	\contribution{潘佳奇}
	
\end{definition}

\begin{theorem}[笛卡尔符号规则(Descartes' rule of signs)]~

	对于任意多项式 $f(x)$,有 $\var(f) \geq \varrho_+(f)$,且 $\var(f) - \varrho_+(f)$ 一定是偶数。

	\contribution{潘佳奇}
	
\end{theorem}

Descartes' rule of signs 只能求出正根的界,对于负根,令 $g(x) = f(-x)$ 即可。

\begin{theorem}[]~

	$$\var(f) = 0 \iff \varrho_+(f) = 0,$$
	$$\var(f) = 1 \implies \varrho_+(f) = 1.$$

	\contribution{潘佳奇}

\end{theorem}

\begin{theorem}[Budan's theorem]~

	对于多项式 $f$,取 $a, b \in \mathbb{R}$,$a < b$,则有:
	$$
	\begin{aligned}
		\var(f(x + a)) & \geq \var(f(x + b)) \\
		\varrho_{a, b} (f) & \leq \var(f(x + a)) - \var(f(x + b)) \\
	\end{aligned}
	$$

	并且 $\var(f(x + a)) - \var(f(x + b)) - \varrho_{a, b} (f)$ 一定是偶数。

	\contribution{潘佳奇}
	
\end{theorem}

\subsection{算法流程}

\subsubsection{实根分离算法-VAS 法}

\begin{algorithm}[实根分离算法-VAS 法]~

	令没有重根的多项式为 $f(x)$,不妨讨论只求出正根的情况。

	维护 $S$ 为算出区间的集合。同时使用 Descartes' rule of signs 和 Budan's theorem 来计算正实根以及区间实根的存在性。

	初始化时 $S = \emptyset$。

	\textbf{步骤 1.} 先判断特殊情况,当 $f(x)$ 不存在正实数根的时候,将 $S$ 作为返回值结束算法。

	\textbf{步骤 2.} 当 $f(0) = 0$ 时,说明 $0$ 是根,将 $0$ 分割出来并且将多项式除以 $x$,并将 $\{0, 0\}$ 加入 $S$,返回\textbf{步骤 1}。

	\textbf{步骤 3.} 当 $f(x)$ 只有一个根的时候,已经达成实数分离的目标,将 $\{(0, +\infty)\}$ 作为结果返回。

	\textbf{步骤 4.} 此时有一个根据多项式系数得到的数 $b$,粗略地标定根的界。(在此算法里设为 $1$)。

	\textbf{步骤 5.} 对多项式做代入 $x = x' + b$,将 $(b, +\infty)$ 映射到 $(0, +\infty)$,并递归求出 $f(x')$ 的实根分割区间后,还原出对应 $(b, +\infty)$ 上的区间集合 $S'$,并令 $S \leftarrow S \cup S'$。

	\textbf{步骤 6.} 如果在 $(0, b)$ 中有实根,则代入 $x = \frac{b}{1 + x'}$,将 $(0, b)$ 映射到 $(0, +\infty)$,并递归求出 $f(x')$ 的实根分割区间后,还原出对应 $(b, +\infty)$ 上的区间集合 $S'$,并令 $S \leftarrow S \cup S'$。

	\textbf{步骤 7.} 将 $S$ 作为返回值返回。

	\contribution[摘自 Polynomial Real Root Isolation Using Vincent’s
Theorem of 1836 第 143 页,有所改编]{潘佳奇}

\end{algorithm}

\subsubsection{实根分离算法-二分法}

\begin{algorithm}[实根分离算法-二分法]~

    \label{bisection-isolation}

    算法接受一个(不含重根的)多项式 $f(x)=a_nx^n+a_{n-1}x^{n-1}+\dots+a_1x+a_0$,并求解其在 $[0, 1)$ 上的实根分割区间。
    
    首先判定多项式常数项 $a_0$ 是否为 $0$。若是,则 $x=0$ 是多项式的零点,并将多项式除以 $x$,直至常数项不为 $0$。

    对于常数项不为 $0$ 的 $n$ 次多项式 $f(x)$,求出 $g(x) = (x+1)^n \cdot f(\frac{1}{x+1})$。

    假设 $f(x)$ 在 $(0, 1)$ 上有某实根 $x_0$,则 $g(\frac{1}{x_0}-1) = 0$。

    同样地,若 $g(x)$ 在 $(0, +\infty)$ 有某实根 $x_1$,则 $f(\frac{1}{x_1+1})=0$。

    亦即,$f(x) = 0$ 在 $(0, 1)$ 上的实根与 $g(x) = 0$ 在 $(0, +\infty)$ 上的实根一一对应。故可对 $g(x)$ 应用 Descartes' rule of signs 以判定 $f(x) = 0$ 在 $(0, 1)$ 上的实根数量。

    若 Descartes' rule of signs 确定 $f(x) = 0$ 在 $(0, 1)$ 上只有零或一个实根,则算法结束。否则,构造 $g_0(x) = 2^n f(\frac{x}{2})$,$g_1(x) = 2^n f(\frac{x+1}{2})$,它们在 $[0, 1)$ 上的实根分别对应了 $f(x) = 0$ 在 $[0, \frac{1}{2}), [\frac{1}{2}, 1)$ 上的实根,递归求解即可。

    \textbf{步骤1.} 初始化队列 q,包含一个三元组 $(c, k, f(x))$,代表当前求解的多项式 $f(x)$ 在 $[0, 1)$ 上的根对应最初函数 $[\frac{c}{2^k}, \frac{c+1}{2^k})$ 上的根。记输入的多项式为 $f_0(x)$,最初的三元组为 $(0, 0, f_0(x))$。

    \textbf{步骤2.} 如果队列 q 为空则结束算法。否则,从队列 q 中取出一个三元组 $(c, k, f(x))$,判断 $f(x)$ 常数项是否为 $0$,如果是,则将 $[\frac{c}{2^k}, \frac{c}{2^k}]$ 置入答案序列中,并将 $f(x)$ 整体除以 $x$,重复执行这一步直到常数项不为 $0$。

    \textbf{步骤3.} 计算 $V = \var((x+1)^n \cdot f(\frac{1}{x+1}))$。

    \textbf{步骤4.} 若 $V = 0$,丢弃该区间并转到 \textbf{步骤2}。

    \textbf{步骤5.} 若 $V = 1$,将该区间 $[\frac{c}{2^k}, \frac{c+1}{2^k}]$ 加入到答案序列并转到 \textbf{步骤2}。

	\textbf{步骤6.} 若 $V > 1$,二分该区间,将三元组 $(2c, k + 1, 2^n f(\frac{x}{2})), (2c + 1, k + 1, 2^n f(\frac{x+1}{2}))$ 加入队列 q,再转到 \textbf{步骤2}。
	
	\contribution{叶隽希}

\end{algorithm}

\subsection{理论证明}

\begin{proof}[笛卡尔符号规则(Descartes' rule of signs)]~

    考虑多项式 $f(x) = a_nx^n + a_{n-1}x^{n-1} + \dots + a_1x + a_0$。
    若 $a_n \neq 1$,不妨将多项式各项系数除以 $a_n$,得到的新多项式首项系数为 $1$,且实根数与变号数均未改变。
    若 $a_0 = 0$,则一定存在某个 $x^m$,将多项式除以 $x^m$ 后得到新的多项式常数项不为 $0$。新的多项式\textbf{正实根数}与变号数均未改变。
    故不妨假设 $a_n=1$ 且 $a_0 \neq 0$。

    \begin{lemma}[]~
        若 $a_0 < 0$,$\var(f(x))$ 为奇数。
        若 $a_0 > 0$,$\var(f(x))$ 为偶数。
    \end{lemma}

    注意到首项系数为正,则若变号奇数次最后为负,变号偶数次最后为正。

    \begin{lemma}[]~
        若 $a_0 < 0$,$\varrho_+(f(x))$ 为奇数。
        若 $a_0 > 0$,$\varrho_+(f(x))$ 为偶数。
    \end{lemma}

    比较直观的感受是,将函数图像在坐标轴上画出,由于首项系数是 $1$,那么多项式在 $x$ 轴正方向足够远处一定在轴上方(即函数值为正)。

    而 $a_0$ 的正负代表了函数图像在 $y$ 轴附近在坐标轴上方或者下方。

    注意到每有一个正实根出现,函数图像必定会穿过 $x$ 轴一次(暂且不考虑重根),则函数值正负必定改变一次。

    故同样,若有奇数个零点则 $a_0$ 与 $a_n=1$ 异号,若有偶数个零点则 $a_0$ 与 $a_n=1$ 同号。

    具体证明可以考虑应用归纳法,每次给多项式乘上一个因子 $(x-p)$,$p$ 如果为正根便会改变 $a_0$ 的正负性,如果为负根便不影响(根为零的情况不存在)。
    
    \begin{lemma}[]~
        $$
        \var(f(x) \cdot (x-p)) > \var(f(x)) (p > 0)
        $$
    \end{lemma}

    在给出证明之前,注意到对于 $p > 0$,有 $\varrho_+(f(x) \cdot (x-p)) = \varrho_+(f(x)) + 1$

    而对于任意一个多项式 $f(x)$,其总可以分解为 $f(x) = g(x) \cdot (x-x_1) \cdot (x-x_2) \dots (x-x_m)$ 的形式,其中 $x_k > 0$ 且 $g(x)=0$ 无正实根。(即提取出所有正根)

    根据定义,$\varrho_+(f(x)) = m$,而 $\var(g(x)) \geq 0$,由引理归纳可得 $\var(f(x)) \geq m = \varrho_+(f(x))$。

    又由前两条引理可知 $\var(f(x))$ 与 $\varrho_+(f(x))$ 奇偶性相同。综合可知定理成立。
	\qedhere

    \begin{proof}[]~

    考虑对 $\var(f(x))$ 产生贡献的每一个二元组 $(i, j)$,它满足 $i < j, a_ia_j < 0, \forall i < k < j, a_k = 0$。

    注意到 $f(x) \cdot (x-p) = \dots + (a_i -p a_{i+1}) x^{i+1} + \dots$。

    若 $a_i < 0$,则 $a_{i+1} \geq 0$,$a_i -p a_{i+1} < 0$

    若 $a_i > 0$,则 $a_{i+1} \leq 0$,$a_i -p a_{i+1} > 0$

    对于每一个二元组 $(i_p, j_p) (1 \leq p \leq \var(f(x)))$,当 $p>1$ 时显然有 $a_{i_{p-1}}a_{i_p} < 0$,故相应的,设 $f(x) \cdot (x-p) = a_{n+1}'x^{n+1} + \dots + a_1'x + a_0'$,则 $a_{i_{p-1}+1}'a_{i_p}' < 0$。

    而又有 $a_0a_{i_1}>0, a_na_{i_{\var(f(x))}}>0, a_0'a_0<0$,故存在序列 $(a_0', a_{i_1+1}, a_{i_2+1}, \dots, a_{i_{\var(f(x))}}, a_n')$,这一序列中任意相邻两项均为异号,而该序列长度为 $\var(f(x)) + 2$,故显然 $\var(f(x) \cdot (x-p)) \geq (\var(f(x)) + 2) - 1$。

    \end{proof}

    \qedhere

    \contribution{叶隽希}
    
\end{proof}

	由于朴素的牛顿法要求初始点离根较近,所以需要二分法来确定包含根的较小区间。下面提出区间运算的牛顿迭代。使用区间运算可以规避这个二分。

\begin{definition}[区间算术]~
	
	定义区间的加减乘和数除为:
	$$
	[l_1, r_1] + [l_2, r_2] = [l_1 + l_2, r_1 + r_2]
	$$
	$$
	[l_1, r_1] - [l_2, r_2] = [l_1 - r_2, r_1 - l_2]
	$$
	$$
	a = l_1 r_1; b = l_1 r_2 ; c = l_2 r_1; d = l_2 r_2
	$$
	$$
	[l_1, r_1] \times [l_2, r_2] = [\min\{a, b, c, d\}, \max\{a, b, c, d\}]
	$$
	
	\contribution{潘佳奇}
\end{definition}

\begin{algorithm}[区间牛顿法]~
	不妨设初始区间只包含正数。
	
	输入:精度 $\varepsilon$,求根区间 $[l, r]$,函数 $f$。
	
	输出:声明不存在根或者输出满足精度要求的一个根。
	
	令 $g(m, [l, r]) = [\min(m - \frac{f(m)}{d_l}, m - \frac{f(m)}{d_r}), \max(m - \frac{f(m)}{d_l}, m - \frac{f(m)}{d_r})]$。
	
	\textbf{步骤1.} 如果 $l > r$ 说明无解。退出迭代。
	
	\textbf{步骤2.} 如果 $r - l < \varepsilon$,表示已经区间收敛到要求范围内。
	
	\textbf{步骤3.} 对于当前区间 $[l, r]$,通过区间算术算出 $[d_l, d_r] = f'([l, r])$。
	
	\textbf{步骤4.} 令 $m = \frac{l + r}{2}$。
	
	\textbf{步骤5.} 如果 $0 \notin [d_l, d_r]$,则令新区间为 $[l', r'] = [l, r] \cap g(m, [d_l, d_r])$,返回步骤 1 继续迭代。
	
	\textbf{步骤6.} 此时 $0 \in [d_l, d_r]$,则对 $[l, r] \cap g(m, [d_l, 0))$ 和 $[l, r] \cap g(m, (0, d_r])$ 产生的两个区间进行递归。
	
	\contribution[摘自 julia 文档\footnotemark]{潘佳奇}
\end{algorithm}
\footnotetext{\url{https://juliaintervals.github.io/pages/explanations/explanationNewton/}} 

\begin{proof}[区间牛顿法的正确性]~

	对于区间 $x_n = [l, r]$, $m = \frac{l + r}{2}$。则 $x_{n + 1} = m - \frac{f(m)}{f'(x_n)}$。

	对于 $f$ 存在 $x^*$ 使得 $f(x^*) = 0$,归纳证明 $x^* \in x_n$。假设对于 $n$ 成立,不妨设 $x^* < m$,存在 $\xi \in (x^*, m) \in (l, r)$,使得 $f'(\xi) = \frac{f(m) - f(x^*)}{m - x^*}$,即 $x^* = m - \frac{f(m)}{f'(\xi)}$,所以 $x^* \in x_{n + 1}$。


	\contribution{潘佳奇}

\end{proof}

当导函数取值不存在零点的时候,该算法二阶收敛至实根,并且必定收敛至实根分离所得初始区间之内的唯一实根。

\begin{algorithm}[VAS + 区间牛顿迭代]~

	\textbf{步骤1.} 使用 Yun's Algorithm 进行平方因子分解。

	\textbf{步骤2.} 使用 VAS 算法进行实根分离。

	\textbf{步骤3.} 使用区间牛顿迭代计算每个区间中根。

	\contribution{潘佳奇}

\end{algorithm}

由于实根估计并不能准确地到达下界,复杂度和二分法相同。

对于区间牛顿迭代,导函数取值不为 $0$ 才能二阶收敛。然而由于多项式区间取值的上下确界不好计算,所以使用了区间算术,这导致很大一部分情况只有一阶的收敛速度。

这个算法在运行效率上没有很大提升,在实际的实现中表现并不好。但是在运算过程中全部使用了区间,避免了牛顿迭代过程中发散的可能。


	\section{对各算法的进一步优化}
	
	\subsection{精度优化}

	不难注意到,提升算法精度的最大瓶颈在于double类型本身无法精确表达几乎全体实数。考虑到我们的算法中涉及的运算均对有理数封闭,我们在各个算法中运用了 gmp 高精度有理数和整数库进行计算,并利用 gmp 库生成测试用多项式,这理论上允许我们做到对任意精度的根进行数值求解。

\subsection{效率优化}

	高精度有理数会带来巨大的算术开销。目前的三个算法的主要区别在于实根分离的手段不同,而在最终求解实根时全部采用牛顿迭代法。牛顿迭代法在大部分条件下二阶收敛至一个实根,其 gmp 有理数的分子分母也二阶增长,这意味着单步迭代的开销会随迭代次数极速上升,严重限制了求解精度的提升。

	我们在对答案不影响的情况下进行了粗略的约分,降低计算开销。

\begin{algorithm}[粗略约分]~
	
	输入:正有理数的最简形式 $\frac{a}{b}$,要求精度 $2^{-\varepsilon}$,向下近似还是向上近似。
	
	输出:粗略约分后的有理数。
	
	\textbf{步骤1.} 找出最小的 $k$ 使得 $2^k \geq \frac{b}{\lfloor\frac{a}{b}\rfloor + 1}$。
	
	\textbf{步骤2.} 如果 $k \leq \varepsilon + C$,则直接输出 $\frac{a}{b}$。其中 $C$ 是一个不低于 $2$ 的常数。
	
	\textbf{步骤3.} 令 $M = 2^{k - \varepsilon - C}$,令 $a' = \frac{a}{M}, b' = \frac{b}{M}$。
	
	\textbf{步骤4.} 如果要求向下近似,输出 $\frac{a'}{b' + 1}$,否则输出 $\frac{a' + 1}{b'}$。
	
	\contribution{潘佳奇}
	
\end{algorithm}

\begin{proof}[粗略约分]~
	
	设 $a = a' M + c, b = b' M + d$,则 $0 \leq c, d < M$。
	
	则有
	
	$$
	\begin{aligned}
		& \left|\frac{a'M + c}{b'M + d} - \frac{a'}{b'}\right| \\
		= & \left|\frac{(a'M + c)b' - (b'M + d)a'}{(b'M + d)b'}\right| \\
		= & \left|\frac{cb' - da'}{(b'M + d)b'}\right| \\
		\leq & \frac{\max(b', a')M}{b'^2M} \\
		= & \frac{\max(1, \frac{a'}{b'})}{b'} \\
		< & \frac{\max(1, \frac{a + 1}{b})}{b'} \\
		< & \frac{1}{2^{\varepsilon + 1}}
	\end{aligned}
	$$
	
	因为 $\frac{1}{b'} < \frac{1}{2^{\varepsilon + 1}}$,所以 $\frac{a' + 1}{b'}, \frac{a'}{b' + 1}$ 都在容差范围内。
	
	并且有 $\frac{a'}{b' + 1} \leq \frac{a}{b} \leq \frac{a' + 1}{b'}$。
	
	\qedhere
	
	\contribution{潘佳奇}
	
\end{proof}

经过验证,使用粗略约分大幅度地加快了牛顿迭代法的速度。


在 Sturm-Newton 方法的实际验证过程中,注意到牛顿迭代的开销过大(实际上粗略约分对该算法的表现改善程度并不大——),故将步骤4的阈值 $L$ 从 $0.1$ 调整为 $L = \min\{\sqrt e, 10^{-2}\}$ ,增加了实根分离的精度,减少了迭代次数,平衡了两部分开销,优化效果明显。

\contribution{吕敬一}


	\section{数值验证与算法改进}
	
	\subsection{根正态分布的多项式}

为了验证数值算法所求根的准确性,我们考虑先生成多项式的精确实根,再根据实根构造出对应的多项式,最后将所求出的实根与精确实根比较。在多项式的生成、输入输出中,应当特别注意控制系数的误差。以下例子说明了微小误差可能带来的影响:

\begin{example}[Wilkinson 多项式]~

	令 $f(x) = (x-1)(x-2)\dots(x-20)$,则有 $x^{19}$ 的系数为 $-210$。假设这一位在系数计算的时候,有误差 $2^{-50}$,即系数 $x^{19}$ 的系数变为了 $-210 - 2^{-50}$,则对应多项式的根会有将近 $10^{-6}$ 的绝对误差。

	以下根由 matlab 计算,并使用 WolframAlpha 的解方程功能进行验算,保留 $30$ 位有效数字。

	$$
	\begin{aligned}
		x_1&=1.0 \\
		x_2&=2.00000000000000000000000007273 \\
		x_3&=2.99999999999999999999854887259 \\
		x_4&=4.00000000000000000194477430517 \\
		x_5&=4.99999999999999946021669717504 \\
		x_6&=6.00000000000005173499077404847 \\
		x_7&=6.99999999999774186363726442871 \\
		x_8&=8.00000000005302033521539523965 \\
		x_9&=8.9999999992545274542491495908 \\
		x_{10}&=10.0000000067448789047907040528 \\
	\end{aligned}
	$$
	$$
	\begin{aligned}
		x_{11}&=10.9999999587489374253459884266 \\
		x_{12}&=12.000000176306266307312631226 \\
		x_{13}&=12.9999994621561736626061961 \\
		x_{14}&=14.0000011839146762070409774592 \\
		x_{15}&=14.9999981178930569438048981748 \\
		x_{16}&=16.0000021383017452406264960673 \\
		x_{17}&=16.9999983085498826309361187107 \\
		x_{18}&=18.0000008842579304437204465685 \\
		x_{19}&=18.9999997255408001641186061227 \\
		x_{20}&=20.0000000382803121076227530584 \\
	\end{aligned}
	$$

	\contribution{潘佳奇}
\end{example}

经过考虑,在生成测试集时,使用正态分布随机根的数据。

\begin{algorithm}[生成测试多项式]~

	对于多项式的度数,随机一个在 $[1,20]$ 上的整数,记为 $d$。以 $80\%$ 的概率生成带有重根的多项式,如果没有重根,令 $d'=d$,否则使用标准正态分布随机一个实数 $p$,并令 $d'=\max(1,d-\lceil \min(|p|,1)d\rceil)$。对于前 $d'$ 个根,使用标准正态分布获取一个值,对于后 $d-d'$ 个根,分别在前 $d$ 个根中等概率随机选取一个根作为自身的值。
	获取根后,将根排序,并使用多项式乘法计算出这些根对应的多项式。在输出的数据中,所有浮点数保留 $64$ 位有效数字。
	
	时间复杂度:$O(d^2)$。
	
	\contribution{潘佳奇}
	
\end{algorithm}

\begin{conjecture}[]~
	
	对于能够保证生成的多项式精度情况较好,使算法在读入的时候带来的误差对根影响较小。
	
	\contribution{潘佳奇}
\end{conjecture}

其中根据正态分布获取随机数的算法如下:

\begin{algorithm}[Marsaglia 极坐标法]~
	
	设正态分布的均值为 $\mu$ ,方差为 $\sigma ^ 2$ 。
	
	\textbf{步骤1.} 不断地随机二元组 $(s, v)$(其中 $u, v$ 均服从 $(-1, 1)$ 上的均匀分布),直到 $0 < u^2 + v^2 < 1$ 。
	
	\textbf{步骤2.} 令 $s = u^2 + v^2, t = \sqrt{\frac{-2\ln s}{s}}$,返回两个数 $\mu + \sigma t \mu, \mu + \sigma tv$ 作为随机数结果。
	
	时间复杂度:$O(1)$。
	
	\contribution{潘佳奇}
\end{algorithm}

\subsection{形态更全面的多项式}

	由于在后续算法优化中使用了 gmp 库,我们用 gmp 库重写了测试方法,包含在testlib.cpp 和testlib.h中。目前的测试方法支持三种功能:随机生成指定数量的根整数、根为实数的多项式,以及复用上一次生成过的多项式。生成的多项式数量nTests和容差要求e需在代码中修改,而生成模式由cin输入。 
	
	\begin{algorithm}[测试方法(简略)]~
		
		输入:一个整数mode,代表测试模式。$0$ 为复用,$1$为整根多项式,$2$为实根多项式。
		
		\textbf{步骤1.} 若 $\textrm{mode} = 0$ ,则从poly.rawin中读取上一次生成的精确原始多项式,以 gmp 高精度有理数形式存储并输出。否则,进入步骤2。
		
		\textbf{步骤2.} 随机生成一个 $[2, 19]$ 的整数 $n$ ,代表多项式的次数。随后按照mode要求,在 $[-10, 10]$ 范围内随机生成 $n$ 个实根,或是在 $[-100, 100]$ 范围内随机生成 $n$ 个不等整根,将其存储为 gmp 高精度有理数格式。
		
		\textbf{步骤3.} 把生成的根展开为多项式,获得 gmp 高精度有理数格式的多项式系数。将 unsigned long long 格式的根和系数输出至poly.raw中。
		
		\textbf{步骤4.} 将多项式分别传入三种不同实现的 gb\_solve() 方法,获得三组不同结果。将三组结果之间分别与生成的标准根比较,输出超出容差的根组数;再将三组结果彼此互相比较,输出超出容差的根组数。
		
		\textbf{步骤5.} 如果希望下一次测试复用某一组数据,则在poly.raw中找到待用数据,复制进poly.rawin中。
		
		\textbf{步骤6.} 若当前测试多项式数量已达到nTests,则退出;否则跳转至步骤2。
		
		\textbf{算法结束。}
		
		\contribution{叶隽希}
		
	\end{algorithm}


\subsection{测试与改进}

	\subsubsection*{读入误差}

	初步测试之后,注意到的首要问题是三种不同算法所求出实根彼此的误差均在所给定容差范围内,但和标准根对比,其均有相同数量的根超出容差。
	
	\begin{example}[读入误差1]~
		
		在求解如下多项式方程时
		$$
		\begin{aligned}
		& -601.8724797782923587874392978847026824951171875 \\ 
		& - 372.28844745806571836510556749999523162841796875x \\
		& + 202.587299451655582060993765480816364288330078125x^2 \\
		& + 52.03053344897322318729493417777121067047119140625x^3 \\
		& - 16.696989593724016032183499191887676715850830078125x^4 \\
		& + x^5
		\end{aligned}
		$$
		
		测试程序提示标准根与算法2, 4 ,5生成的实根各有两个误差超过容差,而算法2, 4, 5生成的实根彼此没有误差超过容差。具体而言,标准根应当是
		$$
		\begin{aligned}
		& -2.53715102852721452464379581215325742959976196289062500 \\
		& -1.18124041966657156166320419288240373134613037109375000 \\
		& 2.51042032439969942103630273777525871992111206054687500 \\
		& 8.56523949619504598729236022336408495903015136718750000 \\
		& 9.33972122132305848651867563603445887565612792968750000
		\end{aligned}
		$$
		
		算法2, 4, 5给出的实根是
		$$
		\begin{aligned}
		& -2.53715102852721496873300566221587359905242919921875000 \\
		& -1.18124041966657178370780911791371181607246398925781250 \\
		& 2.51042032439969942103630273777525871992111206054687500 \\
		& 8.56523949619507440900179062737151980400085449218750000 \\
		& 9.33972122132303006480924523202702403068542480468750000
		\end{aligned}
		$$
		\contribution{吕敬一}
		
	\end{example}

	对 Sturm-Newton 算法调整了迭代终止条件,强制增加了迭代次数,发现输出结果和标准答案的误差完全没有改变,基本说明gb\_solve()的求解过程并未导致过大的误差。 
	
	由于数据生成中采用了无误差的高精度数,初步判断该误差来源是将生成数据输入gb\_solve()时,由于gb\_solve()要求输入多项式系数为double类型,发生了丢失精度的类型转换。根据先前Wilkinson多项式的例子,在次数较高时,多项式系数的微小扰动也会带来实根明显的误差。考虑到double类型丢失精度不可避免,故认为这并不属于算法涉及的明显缺陷。
	
	\begin{example}[读入误差2]~
		
		在mode = 1下生成了具有如下30个实根的多项式:
		$$
		\begin{aligned}
		& -97,-88,-86,-84,-83,-82,-77,-59,-55,-42,-40,-39,-34,-28,-22,-14,-11, \\
		& 3,24,28,47,52,59,64,66,73,80,88,89,91
		\end{aligned}
		$$
		
		注意到实根中有10的倍数(40和80),展开后的多项式常数项应当为10的倍数。然而输入数据显示double类型的多项式系数中常数项为
		$$
		-113260460306063369995491810216948300274714967801856
		$$ 
		
		显然发生了矛盾。
		\contribution{吕敬一}
	
	\end{example}

	这个例子表明double类型存储多项式系数对于过高的次数或者过大的系数也会产生显著的读入误差,且很难避免。

    % 这里好像逻辑跟先前的文档设计不太一样
    % 应该放到自己算法的下面?
    
    \subsubsection*{舍入误差}

    当我们将测试数据调整为度数不超过 10,精确根为绝对值不超过 100 的整数,容差设置为 $10^{-15}$ 时,发现算法的输出出现了问题。

    \begin{example}[舍入误差]~

        在求解如下多项式方程时:

        \[ 108 -21x +x^2 = 0 \]
        
        算法 \ref{bisection-isolation} 给出了 \verb|double| 类型下的结果:

        \[ x_1 = 8.9999999999999982236431605997495353221893310546875, x_2 = 12\]
        
        误差超过了给定的 $10^{-15}$ 。
       	\contribution{吕敬一}
        
    \end{example}

	由于之前测试之中 $10^{-15}$ 的容差在长期测试下并未出现问题,这一组简单数据却超出了容差,这引起了我们的注意。
	
	起初怀疑是算法精度不足,但在同时增加二分层数、牛顿迭代次数、粗略约分的精度后,算法运行时间虽已数十倍地增长,但给出的结果并未改变。
	
	在经历了更多的调试后我们偶然发现,算法给出的 $x_1$ 与实际的根 $9$ 在IEEE 754标准的 \verb|double| 意义下实际是相邻的。
	
	\begin{myverbatim}
		\verbatiminput{reinterpret.cpp}
	\end{myverbatim}
	
	这一段代码给出的输出为
	
	\begin{verbatim}
		9.00000000000000000000000000000000000000000000000000
		0100000000100010000000000000000000000000000000000000000000000000
		8.99999999999999822364316059974953532218933105468750
		0100000000100001111111111111111111111111111111111111111111111111
	\end{verbatim}
	
	随即我们意识到,问题产生于我们使用的高精度库 gmp,由于我们在计算过程中采用的是高精度有理数,而返回时需要转化成 \verb|double| 类型,故而我们使用了 gmp 库提供的 \verb|get_d()| 方法。然而,查阅资料后发现,这一方法返回时并未四舍五入,而是采用了向零取整。而由于算法原理的原因,我们计算出的当前实根近似值虽然非常接近真实值 $9$,但却始终不到 $9$,向零取整后便变为了现在给出的 $x_1$。采用了自己实现的四舍五入方法后,这一问题便改善了(但没有完全解决)。

\end{document}
